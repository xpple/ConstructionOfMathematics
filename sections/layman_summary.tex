\documentclass[../main.tex]{subfiles}
\graphicspath{{\subfix{../images/}}}
\begin{document}
In this thesis we will construct the real numbers. For doing so we will define the necessary mathematical objects using a foundation of mathematics called Zermelo-Fraenkel set theory. We will use these sets to construct first the natural numbers, integers and rational numbers. We will show how these constructed number systems align with our intuition mathematically by proving the properties they are expected to enjoy, as well as by showing how these number systems are mathematically deemed unique. Finally we will construct the real numbers in three ways. The first two constructions are the most common, whereas the third is nonstandard. We will show each construction in essence achieves the same real numbers. The real numbers are lastly characterised in terms of the properties that naturally arose from the first two constructions.
\end{document}
