\documentclass[../main.tex]{subfiles}
\graphicspath{{\subfix{../images/}}}
\begin{document}
In \textit{Analysis with an Introduction to Proof} the following is said~\cite{Lay2014}:
\begin{quote}
    We begin by assuming the existence of a set $\bbR$, called the set of real numbers, and two operations $+$ and $\cdot$, called addition and multiplication, such that the following apply: [\ldots]
\end{quote}
after which a list of requirements (axioms) in terms of $+$, $\cdot$ and elements of the set $\bbR$ is given. Further on a relation ``$<$'' is introduced, and axioms are imposed on it. This construction is called an axiomatic (or synthetic) construction of the real numbers, because it defines the real numbers as a set in which certain axioms apply. Using these axioms, one can then prove more statements about the real numbers, use those to prove even more, and so on. \\\\
Crucially, one may wonder why the following was not said instead: ``The set of real numbers $\bbR$ is the set that satisfies the following axioms: [\ldots].''. Well, what if no such set exists in the first place? Then the definition of $\bbR$ would not even make sense. Secondly, what if many such sets exist? Then it would not be clear which one of those would be the set $\bbR$, given that the definite article ``the'' imposes the existence of only a single one. \\\\
Both these objections are at the heart of this report. To resolve them, we will explicitly construct a suitable set for the real numbers. Then, instead of assuming, we will be proving that the constructed set satisfies the axioms, whereby we resolve the existence objection. Next, we will show that the resulting set is, in a way, unique. That is, any set satisfying the axioms is in essence the same set. This resolves the uniqueness objection. \\\\
It might seem strange to ponder the existence and uniqueness of the real numbers at all. After all, the calculations one performs in their daily lives all involve the real numbers. Measurements of quantities like weight, length and velocity all have their values in the real numbers. It would seem almost nonsensical to question their existence or uniqueness. What would it even mean if the real numbers were not unique? Yet an important branch of mathematics concerns these and similar questions; foundational mathematics. This report aspires to be an accessible introduction to this field. As the name suggests, this type of mathematics is about the objects at the foundation of various other branches of mathematics. The dependency of other fields of mathematics on foundational mathematics is what makes it inherently vital to all of mathematics. It is for this reason that the real numbers are well worth to study thoroughly. We want to formally assure that the real numbers we think exist also exist mathematically, and have the properties we are used to. \\\\
To construct the real numbers, one must start somewhere. This starting point will be Zermelo-Fraenkel set theory. This will be Chapter~\ref{chap:zermelo_fraenkel_set_theory}. Zermelo-Fraenkel set theory centres around one object; the set. Every mathematical object can then be described in terms of sets. To match our intuitive idea of a set as a collection of things to the mathematical object, we impose a list of axioms on them. This will not only help with our intuition, but it will also formally guide us on how one can and cannot operate with sets. Notably, Zermelo-Fraenkel set theory is a framework built on top of a foundation laid by mathematical logic. As such, we will assume knowledge of some definitions and results from mathematical logic. These will be mostly self-explanatory, but for reference one may consult Appendix~\ref{app:first_order_logic}. \\\\
To construct the real numbers we will in Chapter~\ref{chap:the_natural_numbers_integers_and_rational_numbers} first construct simpler number systems. These are the natural numbers, integers and rational numbers. These are not only convenient for the construction of the real numbers, but also of great individual significance. As for the real numbers, specifically the natural numbers, integers and rational numbers appear on a daily basis in ones life. These number systems therefore also warrant a mathematical formalisation. \\\\
In Chapter~\ref{chap:the_real_numbers} we will construct the real numbers. Actually, we will do it three ways. There is in fact no single way to construct the real numbers, or any number system for that matter. While the natural numbers, integers and rational numbers have a mostly standard construction, the real numbers show much diversity. This is also what makes the construction of the real numbers particularly interesting. The third construction will be a nonstandard construction of the real numbers. After the three constructions, we will prove their equivalence. \\\\
The characterisations of number systems throughout this report will be stated in terms of three notions. Addition, multiplication and order. These are the central operations with which the results will be formulated. The most basic sets in which these operations behave nicely with each other yield an important algebraic structure; the ordered semiring. The ordered semiring is the most fundamental structure that unifies all three operations. The sets we will construct can then be characterised in terms of ordered semiring structures with additional properties. Further structure, like exponentiation, can be added but are not strictly necessary. The precise definitions of the algebraic structures we will use will be stated as we need them. \\\\
This report will be written from a perspective that is familiar with how the numbers and operations behave. That is, we intuitively know what numbers are, and subject to what rules we can perform operations on them. Using these intuitions, we will both define and characterise these operations mathematically. This perspective will help to remove the potential arbitrariness that may arise when presented with the constructions without further comment on the ``why'' aspect. Moreover, we will sometimes write things in quotes. All mathematics in quotes is to be read solely for instructional purposes. It appeals to the intuition of the reader, but is not always mathematically sound. \\\\
We will distinguish the following types of results:
\begin{description}
    \item[Proposition.] A minor result of independent interest.
    \item[Lemma.] A minor result for use in the proof of a more major result.
    \item[Theorem.] A major result.
\end{description}
Most results in this report will be propositions. This is due to the nature of foundational mathematics. Each step of the construction establishes results that are useful throughout all of mathematics and are therefore assigned propositions.
\end{document}
