\documentclass[../main.tex]{subfiles}
\graphicspath{{\subfix{../images/}}}
\begin{document}
In this thesis models of the real numbers will be constructed using a set-theoretic approach. The mathematical foundation we will assume is the first-order theory with equality known as Zermelo-Fraenkel set theory ($\mathbf{ZF}$). From the axioms of $\mathbf{ZF}$, the necessary notions for the construction will be introduced as definitional extensions to the language of $\mathbf{ZF}$. We will show how functions can be defined as relations, which in turn are defined as subsets of Cartesian products. Using this preliminary work we will first construct the natural numbers. We will see how our intuition of using the natural numbers for counting gives rise to the Dedekind-Peano axioms ($\mathbf{PA}$). Defining the natural numbers $\bbN$ as the smallest set satisfying the requirement in the statement of the Axiom of Infinity, we find that they are indeed a model of $\mathbf{PA}$. By using our knowledge of what properties operations relating to the natural numbers should satisfy, we induce algebraic structures, the most basal of which being the ordered semiring. We will see how the successor function and induction are intimately tied to the predecessor function and well-ordering. This structure will be used to prove the uniqueness of $\bbN$; it is the unique well-ordered commutative semiring in which every nonzero element has a predecessor. Next the integers $\bbZ$ are defined as equivalence classes of pairs of the natural numbers. We observe how the integers have additive inverses, which allows for a simple characterisation of them as the unique ordered commutative ring whose positive elements are well-ordered. Using the integers, we will define the rational numbers $\bbQ$ as fractions represented by equivalence classes of pairs of integers. An important result will be that the rational numbers are the smallest ordered field, in that every ordered field has a subfield isomorphic to $\bbQ$. After this, we will turn to constructing the real numbers. Three models will be constructed: the Dedekind reals $\bbR_D$, the Cantor reals $\bbR_C$ and the Schanuel reals $\bbR_S$. The first two constructions both induce a fundamental property of the real numbers: completeness. The third construction is a lesser known construction. We will compare the three constructions and prove they are indeed equivalent. That is, we will prove that any ordered field is Dedekind-complete if and only if it is Archimedean and Cauchy-complete. This property of the real numbers will be used to prove that they are the unique ordered Dedekind-complete field. Lastly some directions for further research will be discussed.
\end{document}
