\documentclass[10pt, a4paper]{report}
\usepackage{fontspec}
\newfontfamily\greekfont[Script=Greek, Path=fonts/]{CMU Serif.ttf}
\newfontfamily\cjkfont[Script=CJK, Path=fonts/]{Noto Serif CJK TC.ttf}
\usepackage{polyglossia}
\setdefaultlanguage{english}
\setotherlanguages{german, greek, chinese}
\usepackage{graphicx}
\usepackage{enumitem}
\usepackage{mathtools, amsthm, amssymb}
\usepackage{color}
\usepackage[justification=centering, hypcap=false]{caption}
\usepackage{tikz}
\usetikzlibrary{arrows.meta}

\usepackage[backend=biber, sorting=nty, bibstyle=numeric]{biblatex}
\addbibresource{bibliography.bib}

% Packages that are best loaded last

\usepackage{hyperref}
\hypersetup{
    pdftitle={Construction of Mathematics},
    pdfauthor={F.M.H. van der Els}
    pdfpagemode=FullScreen,
    bookmarksopen=true,
    breaklinks=true,
    colorlinks=false,
    linkbordercolor=red,
    urlbordercolor=blue,
    citebordercolor=blue,
    filebordercolor=blue,
    pdfborderstyle={/S/U/W 1}
}

\usepackage{subfiles}

\DeclareMathOperator{\id}{i}

\newcommand{\bbN}{\mathbb{N}}
\newcommand{\bbZ}{\mathbb{Z}}
\newcommand{\bbQ}{\mathbb{Q}}
\newcommand{\bbR}{\mathbb{R}}
\newcommand{\bbC}{\mathbb{C}}
\newcommand{\bbK}{\mathbb{K}}

\renewcommand{\labelenumi}{(\alph{enumi})}
\renewcommand{\theenumi}{(\alph{enumi})}

\newtheorem{theorem}{Theorem}[section]
\newtheorem{lemma}[theorem]{Lemma}
\newtheorem{proposition}[theorem]{Proposition}
\theoremstyle{definition}
\newtheorem{definition}[theorem]{Definition}
\newtheorem*{example}{Example}

\DeclareMathOperator{\dom}{dom}
\DeclareMathOperator{\corange}{corange}
\DeclareMathOperator{\codom}{codom}
\DeclareMathOperator{\range}{range}

\newcommand{\relR}{\mathrel{R}}

\DeclareMathOperator{\upp}{upp}
\DeclareMathOperator{\low}{low}

\begin{document}

\pagenumbering{roman}

\begin{titlepage}
    \begin{center}
        \vspace*{6em}

        \makebox[\linewidth]{\framebox[\linewidth + 6em]{%
        \begin{minipage}{\linewidth}
            \centering
            \vspace*{3em}
            {\huge Construction of Mathematics}

            \vspace*{1em}

            {\large An Accessible Introduction to Foundational Mathematics}
            \vspace*{3em}
        \end{minipage}}}

        \vspace*{3em}

        {\large\bfseries F.M.H. van der Els \\[1em] Delft, The Netherlands \\ \today}

        \vspace*{3em}

        {\large Source code available at}

        \vspace*{3em}

        {
            \fboxsep=1em
            \fbox{\href{https://github.com/xpple/ConstructionOfMathematics}{GitHub.com/xpple/ConstructionOfMathematics}}
        }

        \vspace*{1em}

        {\small Commit: SHORTCOMMITHASH}
    \end{center}
\end{titlepage}
\newpage

\setcounter{page}{2}

\chapter*{Summary}\label{chap:summary}
\addcontentsline{toc}{chapter}{Summary}
\subfile{sections/summary}
\newpage

\chapter*{Introduction}\label{chap:introduction}
\addcontentsline{toc}{chapter}{Introduction}
\subfile{sections/introduction}
\newpage

\tableofcontents
\newpage

\pagenumbering{arabic}

\chapter{Zermelo-Fraenkel set theory}\label{chap:zermelo_fraenkel_set_theory}
\subfile{sections/body/zermelo_fraenkel_set_theory}
\newpage

\chapter{The natural numbers, integers and rational numbers}\label{chap:the_natural_numbers_integers_and_rational_numbers}
\subfile{sections/body/the_natural_numbers_integers_and_rational_numbers}
\newpage

\chapter{The real numbers}\label{chap:the_real_numbers}
\subfile{sections/body/the_real_numbers}
\newpage

\addcontentsline{toc}{chapter}{Bibliography}
\printbibliography[title={Bibliography}]

\appendix

\chapter{First-order logic}\label{app:first_order_logic}
\subfile{sections/appendices/first_order_logic}

\end{document}
